\noindent
\fcolorbox{orange}{white}{
\begin{minipage}[t]{242mm}
\vspace{5mm}\section*{\fontsize{48}{20}\selectfont Conclusion} \vspace{-5mm}
\linespread{1.9}\fontsize{36}{20}\sffamily\selectfont
%%%%%%%%%% content %%%%%%%%%%
\ tq can handle various types of data in a uniform manner, especially in the field of materials science.
Adopting the syntax of S-expressions, tq incorporates the binding and node referencing mechanism to represent a graph structure that defines input and output data formats.
Due to its expressive power, users can write a set of rules that reform unstructured data (e.g.CSV) into those of an arbitrary format as they need, such as a tensor format for machine learning.

\vspace{5mm}
\end{minipage} }
